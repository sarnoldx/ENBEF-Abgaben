\section{Potentialformulierung und Eichung}
Die vier Maxwell'schen Gleichungen lauten:
\begin{equation}
\label{gl:max1}
\nabla \times \vec{E} = -\partial_t\vec{B}
\end{equation}
\begin{equation}
\label{gl:max2}
\nabla \times \left(\frac{1}{\mu_0}\vec{B}\right) = \partial_t\varepsilon_0\vec{E} + \vec{J}
\end{equation}
\begin{equation}
\label{gl:max3}
\nabla \cdot \vec{B} = 0
\end{equation}
\begin{equation}
\label{gl:max4}
\nabla \cdot \vec{E} = \frac{\rho}{\varepsilon_0}
\end{equation}
Alle Feldgrößen sind von Ort $\vec{r}$ und Zeit $t$ abhängig. Alternativ lässt sich eine Formulierung mithilfe der Potenziale $\vec{A}$ und $\Phi$ aufschreiben, welche implizit durch die Gleichungen
\begin{equation}
\label{gl:max5}
\vec{E} = -\partial_t\vec{A} - \nabla\Phi,
\end{equation}
\begin{equation}
\label{gl:max6}
\vec{B} = \nabla\times\vec{A}
\end{equation}
gegeben sind.

In der Elektrostatik kann Gleichung \ref{gl:max5} durch $\vec{E} = - \nabla\Phi$ vereinfacht werden, da alle Vorgänge statisch sind, also zeitlich unveränderlich. Es gilt $\partial_t\vec{A} = \vec{0}$. Demnach spielen für die Elektrostatik nur Eichfelder $\vec{A}$, $\Phi$ eine Rolle, die zeitunabhängig sind.

Wir betrachten im folgenden nun jedoch den allgemeinen, zeitlich veränderlichen Fall und gehen nun näher auf die Beschreibung der Maxwell'schen Gleichungen durch die Eichfelder $\vec{A}$ und $\Phi$ ein.

Gleichung \ref{gl:max6} impliziert Gleichung \ref{gl:max3}, da man durch Skalarmultiplikation mit dem Nabla-Operator $\nabla$

\begin{align*}
\nabla \cdot \vec{B} &= \nabla \cdot (\nabla\times\vec{A}) \\
\nabla \cdot \vec{B} &= 0
\end{align*}

erhält. Die Schreibweise $\nabla \cdot (\nabla\times\vec{A})$ ist hierbei äquivalent zur Schreibweise $\operatorname{div} \operatorname{rot} \vec{A}$, und demnach immer Null.

Ebenso impliziert Gleichung \ref{gl:max5} die Gleichung \ref{gl:max1}. Hierzu bildet man das Vektorprodukt mit dem Nabla-Operator $\nabla$ auf beiden Seiten der Gleichung:

\begin{align*}
\nabla \times \vec{E} &= \nabla \times (-\partial_t\vec{A} - \nabla\Phi) \\
 &= \nabla \times (-\partial_t\vec{A}) - \nabla \times(\nabla\Phi) \\
 &= \nabla \times (-\partial_t\vec{A}) \\
 &= -\partial_t (\nabla \times \vec{A}) \\
 &= -\partial_t \vec{B} \\
\end{align*}

Angewendet wurde die Rechenregel $\nabla \times(\nabla\Phi) = \operatorname{rot}\operatorname{grad}\Phi = \vec{0} $, ebenso wurde die Gleichung \ref{gl:max6} im letzten Schritt eingesetzt, um wieder auf $\vec{B}$ zu kommen.

