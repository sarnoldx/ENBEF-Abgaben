\chapter{Anhang}\label{sec:anhang}
\lstset{ % Octave Settings
	language=Octave,
	extendedchars=true,
	basicstyle=\footnotesize,
	numbers=left,
	numberstyle=\tiny\color{gray},
	stepnumber=1,
	numbersep=10pt,
	showspaces=false,
	showstringspaces=false,
	tabsize=2,
	breaklines=true,
	frame=single,
	morecomment = [l][\itshape\color{blue}]{\%},
	captionpos=b,
	title=\lstname
}
\begin{equation*}
	G=
	\begin{bmatrix}
	-1& 1&	0&	0&	0&	0&	0&	0\\
	0&	-1&	1&	0&	0&	0&	0&	0\\
	0&	0&	-1&	1&	0&	0&	0&	0\\
	0&	0&	0&	-1&	1&	0&	0&	0\\
	0&	0&	0&	0&	-1&	1&	0&	0\\
	0&	0&	0&	0&	0&	-1&	1&	0\\
	0&	0&	0&	0&	0&	0&	-1&	1\\
	0&	0&	0&	0&	0&	0&	0&	-1\\
	-1&	0&	1&	0&	0&	0&	0&	0\\
	0&	-1&	0&	1&	0&	0&	0&	0\\
	0&	0&	-1&	0&	1&	0&	0&	0\\
	0&	0&	0&	-1&	0&	1&	0&	0\\
	0&	0&	0&	0&	-1&	0&	1&	0\\
	0&	0&	0&	0&	0&	-1&	0&	1\\
	0&	0&	0&	0&	0&	0&	-1&	0\\
	0&	0&	0&	0&	0&	0&	0&	-1\\
	-1&	0&	0&	0&	1&	0&	0&	0\\
	0&	-1&	0&	0&	0&	1&	0&	0\\
	0&	0&	-1&	0&	0&	0&	1&	0\\
	0&	0&	0&	-1&	0&	0&	0&	1\\
	0&	0&	0&	0&	-1&	0&	0&	0\\
	0&	0&	0&	0&	0&	-1&	0&	0\\
	0&	0&	0&	0&	0&	0&	-1&	0\\
	0&	0&	0&	0&	0&	0&	0&	-1\\
	\end{bmatrix}
\end{equation*}
Beispiel eines Divergenzoperators mit $N_x = N_y = N_z = 2$
\lstinputlisting{data/fit_dual_div.m}
\lstinputlisting{data/calc_steps.m}
\lstinputlisting{data/SkriptAg7_1.m}


