\chapter{Bearbeitung der Aufgaben}
\section{Doppelwirbelgleichung in 2D und 3D}

Aus der statischen Doppelwirbelgleichung
\begin{equation}
\label{gl:dw}
\operatorname{rot}\operatorname{rot}\vec{A} = \mu\vec{J},
\end{equation}
lässt sich die skalare Laplace-Gleichung (\ref{gl:laplace})  für den 2D Fall mit $\vec{A} = [0,0,A_z]^T$ herleiten:
\begin{align*}
\operatorname{rot}\operatorname{rot}\vec{A} &= \mu\vec{J} \\
\operatorname{grad}\operatorname{div}\vec{A} - \Delta\vec{A} &= \mu\vec{J},
\end{align*}
wobei
\begin{align*}
\operatorname{grad}\operatorname{div}\vec{A} - \Delta\vec{A} = \nabla\left(\nabla \cdot [0,0,A_z]^T \right)
= \nabla (\partial_z A_z) = 0.
\end{align*}
Hierbei ist $\partial_z A_z = 0$, da $A_z$ im zweidimensionalen Fall nur von $x$ und $y$ anhängt. Damit ergibt sich schließlich die Laplace-Gleichung

\begin{equation}
\label{gl:laplace}
\Delta\vec{A} = -\mu\vec{J}
\end{equation}

Das Feld $\vec{A} = [0,0,A_z]^T$ genügt mit $\nabla \cdot \vec{A} = 0$ der Coulomb-Eichung. Daher ist $\vec{A}$ bis auf eine additive Konstante $A = [0,0,A_z+c]^T$ bestimmt.

Nun soll mithilfe des Faraday'schen Gesetzes (\ref{gl:fara}) die gedämpfte Wellengleichung (\ref{gl:welle}) hergeleitet werden. Hierfür werden die folgenden Maxwell'schen Gleichungen benötigt:

\begin{equation}
\label{gl:fara}
\operatorname{rot}\vec{E} = -\partial_t\vec{B}
\end{equation}
\begin{equation}
\label{gl:max2}
\operatorname{rot}\vec{B} = \mu_0\left(\partial_t\varepsilon_0\vec{E} + \vec{J}\right) \qquad \text{mit } \vec{J} = \sigma\vec{E}
\end{equation}
\begin{equation}
\label{gl:max3}
\operatorname{div}\vec{E} = \frac{\rho}{\varepsilon_0}
\end{equation}

Nach Anwenden des Rotationsoperators auf (\ref{gl:fara}) erhält man:

\begin{align*}
\operatorname{rot}\vec{E} &= -\partial_t\vec{B} \\
\operatorname{rot}\operatorname{rot}\vec{E} &= -\partial_t\operatorname{rot}\vec{B} \\
\operatorname{rot}\operatorname{rot}\vec{E} &= -\partial_t\mu_0(\partial_t\varepsilon_0\vec{E}+\vec{J}) \qquad \text{nach einsetzen von (\ref{gl:max2})} \\
\operatorname{grad}\operatorname{div}\vec{E} -\Delta\vec{E} &= -\partial_t\mu_0(\partial_t\varepsilon_0\vec{E}+\sigma\vec{E}) \\
\operatorname{grad}\left(\frac{\rho}{\varepsilon_0}\right) -\Delta\vec{E} &= -\partial_t\mu_0(\partial_t\varepsilon_0\vec{E}+\sigma\vec{E}) \qquad \text{nach einsetzen von (\ref{gl:max3})}\\
\end{align*}
Wir haben $\rho = 0$ gesetzt, da wir keinen anderen Weg gefunden haben den letzten störenden Term zu eliminieren. Somit gilt unsere gedämpfte Wellengleichung nur im Ladungsfreien Raum.

\begin{equation}
\label{gl:welle}
-\Delta\vec{E} + \mu_0\varepsilon_0\partial_t^2\vec{E} + \underbrace{\mu_0\partial_t(\sigma\vec{E})}_{\text{Dämpfung}} = 0
\end{equation}

Möchte man diese Formel nun mit der Potentialformulierung $\vec{E} = -\partial_t\vec{A} - \nabla\Phi$ ausdrücken erhält man:

\begin{equation}
\Delta(\partial_t\vec{A} + \nabla\Phi) - \mu_0\varepsilon_0\partial_t^2(\partial_t\vec{A} + \nabla\Phi) - \mu_0\partial_t(\sigma(\partial_t\vec{A} + \nabla\Phi)) = 0
\end{equation}

Es wird schnell deutlich, dass dieses Vorgehen nicht empfehlenswert ist. Es handelt sich, abgesehen von Länge der Formel, nun auch um eine Differenzialgleichung dritter Ordnung, zuvor war die Ordnung nur zwei. Dies erschwert das Lösen der Gleichung erheblich.
 