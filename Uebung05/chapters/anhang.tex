\chapter{Anhang}\label{sec:anhang}
\lstset{ % Octave Settings
	language=Octave,
	extendedchars=true,
	basicstyle=\footnotesize,
	numbers=left,
	numberstyle=\tiny\color{gray},
	stepnumber=1,
	numbersep=10pt,
	showspaces=false,
	showstringspaces=false,
	tabsize=2,
	breaklines=true,
	frame=single,
	morecomment = [l][\itshape\color{blue}]{\%},
	captionpos=b,
	title=\lstname
}

\lstinputlisting{data/femmcapacity.m}
\lstinputlisting{data/spherecapacity.m}
\lstinputlisting{data/Ag5_2b.m}
\lstinputlisting{data/Ag5_3b.m}

\subsection*{Methode bisektion}
\begin{lstlisting}[caption={Methode \texttt{bisektion} in Octave. Falls \texttt{a} und \texttt{b} unterschiedliche Vorzeichen haben, konvergiert die Methode gegen eine der Nullstellen aus diesem Intervall. Durch \texttt{maxn} wird die Anzahl der Iterationsschritte und somit auch die Abbruchbedingung definiert}, label=list:bisektion]
function x = bisektion(f, a, b, maxn)
	if(sign(f(a))*sign(f(b)) != -1)
		% Error case (ungueltige Eingabe von a und b)
		x = -777;
		return;
	else
		c = (a+b)/2;
		for n = 1:maxn
			c = (a+b)/2;
			if(f(c) == 0)
				x = c;
				return;
			end
			if(sign(f(a))*sign(f(c)) == -1)
				b = c;
			else
			a = c;
			end
		end
		x = c;
	end
end
\end{lstlisting}

\subsection*{Numerische Lösung Aufgabe 5.3d}
\begin{lstlisting}[caption={Numerische Lösung der Aufgabe 5.3d, durch Umwandlung in ein Nullstellen-Problem}, label=list:ag5.3d]
C = 5*10^(-12);
f = @(R1) ((spherecapacity(R1, 1) - C));
a=1;
b=10;
maxn = 20;
x = bisektion(f, a, b, maxn)
l = f(3.1130)
\end{lstlisting}
