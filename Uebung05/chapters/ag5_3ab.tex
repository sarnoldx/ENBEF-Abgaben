\section{Aufgabe 5.3}
Wie in der vorherigen Aufgabe und den bisherigen Ausarbeitungen gezeigt, lassen sich mit Hilfe von FEMM sehr gute Simulationen von Plattenkondensatoren erzeugen. Aber auch Kugelkondensatoren lassen sich mit FEMM simulieren und berechnen. \\
Die Methode \texttt{spherecapacity.m}, die im Anhang zu finden ist, nimmt die zwei Parameter $R_1$, der den Innenradius beschreibt und $\varepsilon_r$, das die relative Permittivität des Dielektrikums darstellt, entgegen. Aus der Aufgabenstellung geht hervor, dass für den zweiten zur Berechnung der Kapazität benötigten, Parameter $R_2 = R_1 + \SI{7}{\centi\meter}$ gilt. $R_2$ ist der Außenradius des Kugelkondensators. Optional könnte man die Methode auch so schreiben, dass $R_2$ übergeben werden kann.\\
Im Vergleich zu den bisherigen Simulationen in FEMM haben wir nun kein planares Problem mehr, sondern ein achsensymmetrisches. In der Problemdefinition in Zeile 6 wird deshalb ein \texttt{'axi'} übergeben. Damit wird bestimmt, dass sich die erzeugte Fläche in Abbildung \ref{fig:KK} nicht in die Tiefe entwickelt, sondern um die z-Achse gedreht wird. Dadurch erhalten man eine 3-Dimensionale Kugel, die FEMM zwar nicht darstellen, aber berechnen kann.\\ \\
In den Zeilen 12 bis 26 werden zwei Halbkreise erzeugt, der innere Halbkreis wird auf \SI{1}{\volt} geladen, am äußeren Halbkreis liegt eine Spannung von \SI{0}{\volt} an. Des Weiteren werden zwischen Zeile 28 und 31 zwei Blocklabel erstellt, das erste Label liegt zwischen den beiden Halbkreisen und beinhalten die Informationen über die relative Permittivität des Dielektrikums. Mit dem zweiten Label wird sichergestellt, dass nur die von den beiden Halbkreisen und deren Verbindungslinien eingeschlossene Fläche zur Simulation genutzt wird. Hierzu wird dem Label die Property \texttt{<No Mesh>} übergeben. Schließlich werden in Zeile 38 und 39 die Verbindungslinien zwischen den Halbkreisen erstellt, um eine abgeschlossene Fläche zu erzeugen. \\ \\
\begin{figure}[h]
	\centering
	\includegraphics[width=.45\textwidth]{data/Kugelkondensator}
	\caption{Ein mit Hilfe der Routine \texttt{spherecapacity} erzeugter Kugelkondensator mit $R_1 = \SI{7}{\centi\meter}$ }
	\label{fig:KK}
\end{figure}

Mit 
\begin{equation}
	C = 4\pi\varepsilon_0\varepsilon_r\frac{R_2R_1}{R_2-R_1}
	\label{eq:Kapa}	
\end{equation}
lässt sich die Kapazität eines homogenen Kugelkondensators berechnen.
Um den Einfluss der Radien auf die Kapazität zu untersuchen, wurden der Methode \texttt{spherecapacity.m} 100 verschiedene Radien $R_1 \in \mathbb{N}, R_1 \in [1,100]$, mit der Einheit \si{\centi\meter} übergeben. Für den Radius $R_2$ gilt weiterhin
\begin{equation}
	R_2 = R_1 + \SI{7}{\centi\meter}.
	\label{eq:R2}
\end{equation}
Die berechneten Kapazitätswerte sind in Abbildung \ref{fig:Kapa} graphisch dargestellt. Die Formel 
\begin{equation}
	C = 4\pi\varepsilon_0\varepsilon_r\frac{(R_1+\SI{7}{\centi\meter})R_1}{R_1+\SI{7}{\centi\meter-R_1}}
\end{equation}
ergibt sich indem man (\ref{eq:R2}) in (\ref{eq:Kapa}) einsetzt. Durch weiteres vereinfachen folgt, dass die Kapazität
\begin{equation}
	C = 4\pi\varepsilon_0\varepsilon_r\frac{R_1^2+\SI{7}{\centi\meter}}{\SI{7}{\centi\meter}}
\end{equation}
quadratisch von $R_1$ abhängt, daraus ergibt sich ein quadratischer Verlauf der Kapazität des Kugelkondensators mit steigendem Innenradius.

\begin{figure}
	\centering
	\includegraphics[width=\textwidth]{data/Kugelkapa}
	\caption{Entwicklung der Kapazität mit steigendem Radius $R_1 \in \mathbb{N}, R_1 \in [1,100]$, $[R_1] = \centi\meter ~$ und $R_2 = R_1 + \SI{7}{\centi\meter}$}
	\label{fig:Kapa}
\end{figure}

Unter der Voraussetzung, dass $(R_2+R_1) = \SI{15}{\centi\meter}$ gilt, ergibt sich für den Innendurchmesser $R_1 = \SI{4}{\centi\meter}$ und für den Außendurchmesser $R_2 = \SI{11}{\centi\meter}$. Analytisch lässt sich mit (\ref{eq:Kapa}) eine Kapazität $C_{\mathrm{ana}} = \SI{6,994}{\pico\farad}$ berechnen, FEMM berechnet eine Kapazität $C_{\mathrm{num}} = \SI{6,986}{\pico\farad}$. \\
Der relative Fehler
\begin{equation}
	\mathrm{err :=} \frac{|C_{\mathrm{ana}}-C_{\mathrm{num}}|}{|C_{\mathrm{ana}}|}
	\label{eq:err}
\end{equation}
ergibt sich mit den vorher berechneten Werten $C_{\mathrm{ana}}$ und $C_{\mathrm{num}}$ zu $1.1 \cdot 10^{-3}$, liegt also im niedrigen Promillebereich, die durch FEMM berechnete Annäherung ist also sehr exakt.