Bestimmt werden soll nun, für welche Werte $R_1$ und $R_2$ die Kapazität den Wert
$C = \SI{5}{\pico\farad}$ an nimmt.

\subsection*{Analytische Betrachtung}
Es gelten folgende Zusammenhänge:

\begin{equation}
\SI{5}{\pico\farad} = 4\pi\varepsilon_0\varepsilon_r\frac{R_2R_1}{R_2-R_1} \text{, sowie } 
R_2 = R_1 + \SI{7}{\centi\meter}
\end{equation}

Einsetzen der zweiten Formel liefert eine quadratische Gleichung für $R_1$

\begin{equation}
\SI{5}{\pico\farad} = 4\pi\varepsilon_0\varepsilon_r\frac{R_1(R_1+\SI{7}{\centi\meter})}{\SI{7}{\centi\meter}}
\end{equation}

Für die man Folgendes Ergebnis erhält:

\begin{equation}
R_1 = \SI{-7/2}{\centi\meter} \pm \sqrt{\left(\SI{7/2}{\centi\meter}\right)^2 + \dfrac{\SI{7}{\centi\meter} \cdot \SI{5}{\pico\farad}}{4\pi\varepsilon_0\varepsilon_r}}
\end{equation}

Dabei ist nur das Ergebnis $R_1 = \SI{3.1111}{\centi\meter}$ positiv und in diesem Zusammenhang sinnvoll.

\subsection*{Numerische Betrachtung}

Um das Problem numerisch zu untersuchen, wurde zunächst das Bisektionsverfahren (siehe Listing \ref{list:bisektion}) in Octave implementiert. Dieses erlaubt durch wiederholte Intervallhalbierung Nullstellen komplizierter Funktionen näherungsweise zu bestimmen. Die Wahl der Startpunkte ist hierbei für die Konvergenz entscheidend. Voraussetzung ist, dass beide Funktionswerte zu den Startwerten $a$ und $b$ unterschiedliche Vorzeichen haben und die gesuchte Nullstelle auch in diesem Intervall liegt. Durch vorherige Abschätzungen wurden die Startwerte $a=1$ und $b=10$ für den Methodenaufruf gewählt (siehe Listing \ref{list:ag5.3d}).

Die numerische Berechnung ergab die Lösung $R_1 = \SI{3.1130}{\centi\meter}$. Das entspricht einer Abweichung von $\SI{19}{\micro\meter}$ zu dem zuvor berechneten Wert.

\subsection*{Fehlerbetrachtung}

Die analytische Lösung ist exakt. Folglich entstehen auf numerischem Weg Ungenauigkeiten, die aus Rundungsfehlern resultieren. Besonders das Rechnen mit sehr großen und sehr kleinen Zahlen gleichzeitig kann unter Umständen am Computer zu Problemen führen, wenn Basisoperationen mit nicht ausreichender Stellenanzahl durchgeführt werden. Fehleranfällig kann jedoch auch schon die Simulation in FEMM sein. Da dort das Problem nur durch ein diskretes, also nicht exaktes, Modell simuliert wird.

Jedoch lässt sich auch in der Realität ein Versuchsaufbau nicht Fehlerfrei gestalten. So können dort Störeffekte durch bspw. externe Magnetfelder oder geringfügig inhomogene Dielektrika auftreten. Zudem ist auch einfach die Messgenauigkeit der Instrumente beschränkt. 