\section*{Einleitung}\label{sec:intro}
\addcontentsline{toc}{section}{Einleitung}
Diese Arbeit beschäftigt sich mit dem Übungsblatt 2 des Faches \dq Einführung in die numerische Berechnung elektromagnetischer Felder\dq{}. Zunächst wird eine analytische Berechnung des dynamischen Verhaltens eines harmonischen Oszillators durchgeführt, indem die sich aus den Maschengleichungen ergebenden Differentialgleichungen gelöst werden. Anschließend wird das dynamische Verhalten des Oszillators numerisch berechnet und mit den analytischen Ergebnissen verglichen. Zur numerischen Simulation wird die Software \dq LT-Spice\dq{} verwendet. In der letzten Aufgabe wird die in LT-Spice aufgebaute Schaltung in Matrizenform umgewandelt. Diese Umwandlung wird durch eine Routine in der Octave-Software durchgeführt. Die Schaltung wird anschließend durch das BDF-Zeitintegrationsverfahren $dassl$ in Octave gelöst.