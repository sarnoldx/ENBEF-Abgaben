\section*{Anhang}
\lstset{ % Octave Settings
	language=Octave,
	extendedchars=true,
	basicstyle=\footnotesize,
	numbers=left,
	numberstyle=\tiny\color{gray},
	stepnumber=1,
	numbersep=10pt,
	showspaces=false,
	showstringspaces=false,
	tabsize=2,
	breaklines=true,
	frame=single,
	morecomment = [l][\itshape\color{blue}]{\%},
	captionpos=b,
	title=\lstname
}
\subsection*{Methode circuit\_matrices}
\begin{lstlisting}[caption={Methode \texttt{circuit\_matrices} in Octave}, label=circuit_matrices]
function [AC,AL,AR,AV,AI,C,L,R,V,I] = circuit_matrices(filename)

% PEMCE Uebung 2: Routine um Schaltunsmatrizen zu erzeugen:
%  [AC,AL,AR,AV,AI,C,L,R,V,I] = circuit_matrices(filename)
%
% Eingabe:
%  filename  - Dateiname der Netzliste
%
% Ausgaben:
%  AC,...,AI - reduzierte Inzidenzmatrizen fuer Kapazitaten, ..., Stromquellen
%  C,...,I   - Matrizen der Parameter (Kapazitaten, ..., Stromquellen)
	
	%# lese Schaltung aus netzliste
	cirstruct = prs_spice (filename);
	
	%# initialisiere variablen
	n = cirstruct.totextvar;
	AC=AL=AR=AV=AI=sparse(n,0);
	C=L=R = zeros(0,0);
	V=I= zeros(0,1);
	
	% Durchlaufe alle Elementtypen (R,L,C,...)
	for i=1:size(cirstruct.LCR,2)
		typ=cirstruct.LCR(i).parnames{1,1};
		sz=size(cirstruct.LCR(i).pvmatrix,1);
	
		% Durchlaufe alle Bauteile eines Typen
		for j=1:size(cirstruct.LCR(i).pvmatrix,1)
			val=cirstruct.LCR(i).pvmatrix(j,:);
			pins=cirstruct.LCR(i).vnmatrix(j,:);
			
			if typ == 'L'
				matrix = AL;
				diagonal = L;
			endif 
			if typ == 'R'
				matrix = AR;
				diagonal = R;
			endif 
			if typ == 'C'
				matrix = AC;
				diagonal = C;
			endif 
			if typ == 'V'
				matrix = AV;
				diagonal = V;
			endif  
			if typ == 'I'
				matrix = AI;
				diagonal = I;  
			endif
			
			is_voltage_or_current = typ == 'V' || typ == 'I';
			
			matrix = [matrix,sparse(n,1)]
			%Diagonalmatrix Elemente einfuegen
			
			if is_voltage_or_current == false
				diagonal = [diagonal,zeros(sz,1)];
			
				if typ == 'R'
					val = 1/val;
				endif
			
				diagonal(j,j) = val;
			endif
			
			%Richtung des Stroms bestimmen
			direction = pins(1) - pins(2);
			
			%Strom fliesst von 1 nach 2
			if direction < 0
				%Nur hinzufuegen wenn es sich nicht um den Masseknoten handelt
				if pins(1) != 0
				matrix(pins(1),j)=1;
			endif
			matrix(pins(2),j)=-1;
			
			if is_voltage_or_current == true
				diagonal = [diagonal;val];
			endif
			
			%Strom fliesst von 2 nach 1
			else
				matrix(pins(1),j)=1;
				%Nur hinzufuegen wenn es sich nicht um den Masseknoten handelt
				if pins(2) != 0
					matrix(pins(2),j)=-1;
				endif
			
				if is_voltage_or_current == true
					diagonal = [diagonal;-val];
				endif
			endif
			
			if typ == 'L'
				AL = matrix;
				L = diagonal;
			endif 
			if typ == 'R'
				AR = matrix;
				R = diagonal;
			endif  
			if typ == 'C'
				AC = matrix;
				C = diagonal;
			endif  
			if typ == 'V'
				AV = matrix;
				V = diagonal;
			endif 
			if typ == 'I'
				AI = matrix;
				I = diagonal;  
			endif
		
		endfor
	endfor
endfunction
\end{lstlisting}
\subsection*{Methode calculate\_matrices}
\begin{lstlisting}[caption={Methode \texttt{calculate\_matrices} in Octave}, label=calculate_matrices]
function [M,K,r] = calculate_matrices(filename)
	
%Berechnet die Matrizen M,K und r  
	[AC,AL,AR,AV,AI,C,L,R,V,I] = circuit_matrices(filename);
	bl = size(L,1);
	bv = size(V,1);
	n = size(AR,1);
	
	M = [AC*C*AC' zeros(n,bl) zeros(n,bv);
	zeros(bl,n) L zeros(bl,bv);
	zeros(bv,n) zeros(bv,bl) zeros(bv,bv)];
	
	K = [AR*R*AR' AL AV;
	-AL' zeros(bl,bl) zeros(bl,bv);
	-AV' zeros(bv,bl) zeros(bv,bv)];
	
	r = [-AI*I; zeros(bl,1); -V];
endfunction
\end{lstlisting}
\subsection*{Finales Skript zum Berechnen der Differenzialgleichung}
\begin{lstlisting}[caption={\texttt{FinalesSkript} in Octave}, label=finalesSkript]
t = [0:1e-5:0.003];

% Bestimmung des Gleichungssystems:
[M,K,r] = calculate_matrices('oszilator2.1.net');
res = @(y, yd,t) (M*yd+K*y-r);

% Bestimmung der Anfangswerte:

% x0:
% phi_1 = 0
% phi_2 = 12
% i_L   = 0
U_c = 12;
L = 0.00173007;
R = 2;
x0 = zeros(size(r));
x0(2) = U_c;

% x0_Strich: 
% phi_1' = 13872,27
% phi_2' = 0
% i_L'   = -6936,16
x0_Strich = zeros(size(r));
x0_Strich(1) = R*U_c/L;
x0_Strich(3) = -U_c/L;

% Nummerische Berechnung des Gleichungssystems:
x = dassl(res, transpose(x0), zeros(size(r)), t);

% Erstellen des Plots:
plot(t,x(:,1:3));
xlabel('Zeit t in [s]', 'interpreter', 'tex')
ylabel('Strom in [A], Spannung in [V]', 'interpreter', 'tex')
legend('\phi_1', '\phi_2', ' i_L')
\end{lstlisting}