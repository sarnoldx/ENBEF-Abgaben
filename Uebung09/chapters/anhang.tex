\section{Anhang}
\lstset{ % Octave Settings
	language=Octave,
	extendedchars=true,
	basicstyle=\footnotesize,
	numbers=left,
	numberstyle=\tiny\color{gray},
	stepnumber=1,
	numbersep=10pt,
	showspaces=false,
	showstringspaces=false,
	tabsize=2,
	breaklines=true,
	frame=single,
	morecomment = [l][\itshape\color{blue}]{\%},
	captionpos=b,
	title=\lstname
}

\subsection*{Skript Aufgabe 9.3}
\begin{lstlisting}[caption={Berechnung der Aufgaben 9.3 a) bis g)}, label=ag3]
clear;
Nx = 2;
Ny = 9;
Nz = 9;
Np = Nx*Ny*Nz;

xmesh = [0:20:20];
ymesh = [0:5:40];
zmesh = [0:2.5:20];

nx = sparse(Np,1);
ny = sparse(Np,1);
nz = sparse(Np,1);

indxOmega1 = sparse(3*Np,1);
indxOmega2 = sparse(3*Np,1);
stepY = calc_steps(Ny,40);
stepZ = calc_steps(Nz,20);
h = calc_steps(Nx,20);


n = 1;
for z = 1 : Nz
for y = 1 : Ny
for x = 1 : Nx
if (z*stepZ >= 7.5 && z*stepZ <= 12.5)
if (y*stepY >= 10 && y*stepY <= 15)
indxOmega1(n) = 1;
end
if (y*stepY >= 25 && y*stepY <= 30)
indxOmega2(n) = -1;
end

end
if (x < Nx) && ((y == 1) || (y == Ny) || (z == 1) || (z == Nz))  
nx(n) = 1;
end

if (y < Ny) && ((x == 1) || (x == Nx) || (z == 1) || (z == Nz))  
ny(n) = 1;
end

if (z < Nz) && ((x == 1) || (x == Nx) || (y == 1) || (y == Ny))  
nz(n) = 1;
end

n = n+1;
end
end
end

indxT = [nx;ny;nz];

j = (h/5)^2*(indxOmega1+indxOmega2);
j = sparse(j);
jr = j;

Mny = createMny(xmesh,ymesh,zmesh,ones(Np,1));
[C,S,Ss] = fit_operator(Nx,Ny,Nz);
K = C'*Mny*C;

%Loesche Rand raus
for i=length(indxT):-1:1
if indxT(i) == 1
K(i,:)=[];
K(:,i)=[];
j(i,:)=[];
end
end

a = K\j;

%Fuege Rand wieder hinzu
ar = sparse(length(indxT),1);
counter = 1;
for i=1:length(indxT)
if indxT(i) == 0
ar(i) = a(counter);
counter = counter +1;
end
end

b = C*ar;

%fit_write_vtk(xmesh, ymesh, zmesh, 'b-Fluss.vtr', {'j',j;'b',b})

L = jr' * ar;
\end{lstlisting}