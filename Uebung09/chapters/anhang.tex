\section{Anhang}
\lstset{ % Octave Settings
	language=Octave,
	extendedchars=true,
	basicstyle=\footnotesize,
	numbers=left,
	numberstyle=\tiny\color{gray},
	stepnumber=1,
	numbersep=10pt,
	showspaces=false,
	showstringspaces=false,
	tabsize=2,
	breaklines=true,
	frame=single,
	morecomment = [l][\itshape\color{blue}]{\%},
	captionpos=b,
	title=\lstname
}

\lstinputlisting{data/fit_operator.m}

\subsection{Skript Aufgabe 9.2}
\begin{lstlisting}[caption={Aufgabe 9.2 Berechnung der Materialmatrix}, label=Matmatrix]
	function Mny = createMny(xmesh,ymesh,zmesh,ny)
	Nx = size(xmesh);
	Nx = Nx(1,2);
	Ny = size(ymesh);
	Ny = Ny(1,2);
	Nz = size(zmesh);
	Nz = Nz(1,2);
	Np = Nx*Ny*Nz;
	Mx = 1;
	My = Nx;
	Mz = Nx*Ny;
	
	Mny = sparse(3*Np,3*Np);
	[ddS, dS, ddA, dA] = fit_calc_elements(xmesh,ymesh,zmesh);
	
	for n = 1 : Np
	if n-Mx>=1
	avgny = fit_calc_avgny(dS(n,1),dS(n-Mx,1),ddS(n,1),ny(n,1),ny(n-Mx,1));
	else
	avgny = fit_calc_avgny(dS(n,1),0,ddS(n,1),ny(n,1),0);
	end
	
	if dA(n,1) == 0
	Mny(n,n) = 0;
	else
	Mny(n,n) = (avgny*ddS(n,1))/dA(n,1);
	end
	end
	
	for n = 1 : Np
	if n-My>=1
	avgny = fit_calc_avgny(dS(n,2),dS(n-My,1),ddS(n,2),ny(n,1),ny(n-My,1));
	else
	avgny = fit_calc_avgny(dS(n,2),0,ddS(n,2),ny(n,1),0);
	end
	
	if dA(n,2) == 0
	Mny(n,n) = 0;
	else
	Mny(Np+n,Np+n) = (avgny*ddS(n,2))/dA(n,2);
	end
	end
	
	for n = 1 : Np
	if n-Mz  >= 1
	avgny = fit_calc_avgny(dS(n,3),dS(n-Mz,3),ddS(n,3),ny(n,1),ny(n-Mz,1));
	else
	avgny = fit_calc_avgny(dS(n,3),0,ddS(n,3),ny(n,1),0);
	end
	
	if dA(n,3) == 0
	Mny(n,n) = 0;
	else
	Mny(2*Np+n,2*Np+n) = (avgny*ddS(n,3))/dA(n,3);
	end
	end
	Mny = sparse(Mny);
	end
\end{lstlisting}

\begin{lstlisting}[caption={Aufgabe 9.2 Berechnung gemittelte Reluktivität}, label=mittel]
	function nymid = fit_calc_avgny(L1,L2,dL1,ny1,ny2)
	nymid = (L1*ny1+L2*ny2)/(2*dL1);
	end
\end{lstlisting}

\begin{lstlisting}[caption={Aufgabe 9.2 Berechnung Kantenlänge und Flächeninhalte}, label=kanten]
	function [ddS,dS,ddA,dA] = fit_calc_elements(xmesh,ymesh,zmesh)
	Nx = size(xmesh);
	Nx = Nx(1,2);
	Ny = size(ymesh);
	Ny = Ny(1,2);
	Nz = size(zmesh);
	Nz = Nz(1,2);
	Np = Nx*Ny*Nz;
	Mx = 1;
	My = Nx;
	Mz = Nx*Ny;
	
	
	stepx = calc_steps(Nx,xmesh(1,Nx)-xmesh(1,1));
	stepy = calc_steps(Ny,ymesh(1,Ny)-ymesh(1,1));
	stepz = calc_steps(Nz,zmesh(1,Nz)-zmesh(1,1));
	
	dA = sparse(Np,3);
	ddA = sparse(Np,3);
	dS = sparse(Np,3);
	ddS = sparse(Np,3);
	
	dx = sparse(Np,1);
	dy = sparse(Np,1);
	dz = sparse(Np,1);
	
	%%%%% Berechnung dS und ddS %%%%%
	n = 1;
	
	for nz = 1 : Nz
	for ny = 1 : Ny
	for nx = 1 : Nx
	if nx < Nx
	dx(n) = stepx;
	end
	if ny < Ny
	dy(n) = stepy;
	end
	if nz < Nz
	dz(n) = stepz;
	end
	n = n+1;
	end
	end
	end
	
	dS = [dx dy dz];
	
	ddx = sparse(Np,1);
	ddy = sparse(Np,1);
	ddz = sparse(Np,1);
	
	z = 1;
	for i = 1 : Np
	
	if (i <= Nx*Ny*z-Nx) && (z < Nz)
	if mod(i,Nx) == 0  
	ddx(i) = dx(i-1)/2;
	elseif mod(i,Nx) == 1 
	ddx(i) = dx(i)/2;
	else
	ddx(i) = dx(i)/2+dx(i-Mx)/2;
	end
	end
	
	if (z < Nz)
	if i < Nx*z
	ddy(i) = dy(i)/2;
	elseif (i < Nx*Ny*z) && (i > (Nx*Ny*z-Nx)) 
	ddy(i) = dy(i-My)/2;
	elseif mod(i,Nx) == 0
	ddy(i) = 0;
	else
	ddy(i) = dy(i)/2+dy(i-My)/2;
	end
	end
	
	if (z <= Nz)
	if (mod(i,Nx) ~= 0) && (i < Nx*(Ny-1))
	ddz(i) = dz(i)/2;
	elseif (mod(i,Nx) ~= 0) && (i < Nx*Ny*z-Nx) && (z < Nz)
	ddz(i) = dz(i)/2+dz(i-Mz)/2;
	elseif (mod(i,Nx) ~= 0) && (i < Nx*Ny*z-Nx)
	ddz(i) = dz(i-Mz)/2;
	else
	ddz(i) = 0;
	end
	end
	
	if mod(i,Nx*Ny) == 0
	z = z+1;
	end
	
	end
	
	ddS = [ddx ddy ddz];
	
	
	%%%%% Berechnung dA und ddA %%%%%
	for i = 1 :  Np
	dA(i,1) = dS(i,2)*dS(i,3);
	dA(i,2) = dS(i,1)*dS(i,3);
	dA(i,3) = dS(i,2)*dS(i,1);
	
	ddA(Np+1-i,1) = ddS(Np+1-i,2)*ddS(Np+1-i,3);
	ddA(Np+1-i,2) = ddS(Np+1-i,1)*ddS(Np+1-i,3);
	ddA(Np+1-i,3) = ddS(Np+1-i,2)*ddS(Np+1-i,1);
	end
	
	end
	
\end{lstlisting}


\subsection*{Skript Aufgabe 9.3}
\begin{lstlisting}[caption={Berechnung der Aufgaben 9.3 a) bis g)}, label=ag3]
clear;
Nx = 2;
Ny = 9;
Nz = 9;
Np = Nx*Ny*Nz;

xmesh = [0:20:20];
ymesh = [0:5:40];
zmesh = [0:2.5:20];

nx = sparse(Np,1);
ny = sparse(Np,1);
nz = sparse(Np,1);

indxOmega1 = sparse(3*Np,1);
indxOmega2 = sparse(3*Np,1);
stepY = calc_steps(Ny,40);
stepZ = calc_steps(Nz,20);
h = calc_steps(Nx,20);


n = 1;
for z = 1 : Nz
for y = 1 : Ny
for x = 1 : Nx
if (z*stepZ >= 7.5 && z*stepZ <= 12.5)
if (y*stepY >= 10 && y*stepY <= 15)
indxOmega1(n) = 1;
end
if (y*stepY >= 25 && y*stepY <= 30)
indxOmega2(n) = -1;
end

end
if (x < Nx) && ((y == 1) || (y == Ny) || (z == 1) || (z == Nz))  
nx(n) = 1;
end

if (y < Ny) && ((x == 1) || (x == Nx) || (z == 1) || (z == Nz))  
ny(n) = 1;
end

if (z < Nz) && ((x == 1) || (x == Nx) || (y == 1) || (y == Ny))  
nz(n) = 1;
end

n = n+1;
end
end
end

indxT = [nx;ny;nz];

j = (h/5)^2*(indxOmega1+indxOmega2);
j = sparse(j);
jr = j;

Mny = createMny(xmesh,ymesh,zmesh,ones(Np,1));
[C,S,Ss] = fit_operator(Nx,Ny,Nz);
K = C'*Mny*C;

%Loesche Rand raus
for i=length(indxT):-1:1
if indxT(i) == 1
K(i,:)=[];
K(:,i)=[];
j(i,:)=[];
end
end

a = K\j;

%Fuege Rand wieder hinzu
ar = sparse(length(indxT),1);
counter = 1;
for i=1:length(indxT)
if indxT(i) == 0
ar(i) = a(counter);
counter = counter +1;
end
end

b = C*ar;

%fit_write_vtk(xmesh, ymesh, zmesh, 'b-Fluss.vtr', {'j',j;'b',b})

L = jr' * ar;
\end{lstlisting}