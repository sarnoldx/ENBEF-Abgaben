\documentclass[
ngerman,
twoside,
pdfa=false,
ruledheaders=section,%Ebene bis zu der die Überschriften mit Linien abgetrennt werden, vgl. DEMO-TUDaPub
class=report,% Basisdokumentenklasse. Wählt die Korrespondierende KOMA-Script Klasse
thesis={type=sta},% Dokumententyp Thesis, für Dissertationen siehe die Demo-Datei DEMO-TUDaPhd
accentcolor=TUDa-2c,% Auswahl der Akzentfarbe
custommargins=false,% Ränder werden mithilfe von typearea automatisch berechnet
marginpar=false,% Kopfzeile und Fußzeile erstrecken sich nicht über die Randnotizspalte
%BCOR=5mm,%Bindekorrektur, falls notwendig
parskip=half-,%Absatzkennzeichnung durch Abstand vgl. KOMA-Sript
fontsize=11pt,%Basisschriftgröße laut Corporate Design ist mit 9pt häufig zu klein
%	logofile=tuda_logo.pdf, %Falls die Logo Dateien nicht installiert sind
]{tudapub}

%%%%%%%%%%%%%%%%%%%%%%%%%%%%
% Download des TU-Logos
%%%%%%%%%%%%%%%%%%%%%%%%%%%%
% https://download.hrz.tu-darmstadt.de/protected/CE/TUDa_LaTeX/tuda_logo.pdf
% Der Pfad zum Logo kann als "logofile" angegeben werden.

%%%%%%%%%%%%%%%%%%%
% Sprachanpassung & Verbesserte Trennregeln
%%%%%%%%%%%%%%%%%%%
\usepackage[english, main=ngerman]{babel}
\usepackage[autostyle]{csquotes}% Anführungszeichen vereinfacht
\usepackage{microtype}

%%%%%%%%%%%%%%%%%%%
% Literaturverzeichnis
%%%%%%%%%%%%%%%%%%%
\usepackage{biblatex}   % Literaturverzeichnis
\addbibresource{HausarbeitBib.bib}

%%%%%%%%%%%%%%%%%%%
% Paketvorschläge Tabellen
%%%%%%%%%%%%%%%%%%%
%\usepackage{array}     % Basispaket für Tabellenkonfiguration, wird von den folgenden automatisch geladen
\usepackage{tabularx}   % Tabellen, die sich automatisch der Breite anpassen
%\usepackage{longtable} % Mehrseitige Tabellen
%\usepackage{xltabular} % Mehrseitige Tabellen mit anpassarer Breite
\usepackage{booktabs}   % Verbesserte Möglichkeiten für Tabellenlayout über horizontale Linien

%%%%%%%%%%%%%%%%%%%
% Paketvorschläge Mathematik
%%%%%%%%%%%%%%%%%%%
\usepackage{mathtools} % erweiterte Fassung von amsmath
\usepackage{amssymb}   % erweiterter Zeichensatz
\usepackage[decimalsymbol=comma]{siunitx}   % Einheiten


%%%%%%%%%%%%%%%%%
% Eigenen Pakete Gruppe03
%%%%%%%%%%%%%%%%%%%%
%\usepackage[utf8]{inputenc}
%\usepackage[ngerman]{babel}
\usepackage{hyperref}
\usepackage{graphicx}
\usepackage{subcaption}
\usepackage{listings}
\usepackage[framed, numbered]{matlab-prettifier}
%\usepackage[style=numeric]{biblatex}
%\usepackage{amsthm}
%\usepackage[squaren]{SIunits}
\usepackage{enumitem}
\usepackage{tikz}
\usepackage{pgfplots}
\usepackage{pgfplotstable}
%\usepackage{booktabs}
\pgfplotsset{compat=1.12}
\usepackage{dsfont}

%%%%%%%%%%%%%%%%%%%
% Pseudocode
%%%%%%%%%%%%%%%%%%%
\usepackage[linesnumbered,lined,boxruled]{algorithm2e} % Package für Pseudocode

%%%%%%%%%%%%%%%%%%%
% Plotting und Grafik
%%%%%%%%%%%%%%%%%%%
\usepackage{tuda-pgfplots} % Package für Plotting with TUDa mods

%%%%%%%%%%%%%%%%%%%
% Sonstiges
%%%%%%%%%%%%%%%%%%%
\usepackage{blindtext} % Package für Blindtext

\begin{document}
	\title{Ausarbeitung Übung 2}
	%\subtitle{Ein Untertitel, wenn nötig}
	\author[D. Schiller, C. Kramer, S.Arnold, T. Lingenberg]{Dominik Schiller \and Constanze Kramer \and Simon Arnold \and Tobias Lingenberg} %optionales Argument ist die Signatur,
	%\reviewer{Gutachter 1 \and Gutachterin 2} %Gutachten
	
	%Diese Felder werden untereinander auf der Titelseite platziert.
	\department{} % Das Kürzel wird automatisch ersetzt und als Studienfach gewählt, siehe Liste der Kürzel im Dokument.

	
	\date{\today}
	%\examdate{\today}
	
	%	\tuprints{urn=1234,printid=12345}
	%	\dedication{Für alle, die \TeX{} nutzen.}
	
	\maketitle
	\pagenumbering{gobble} % Seitenzahlen angezeigt, startet ab dem Inhaltsverzeichnis
	
	
	\affidavit
	%\AffidavitSignature
	%\AffidavitSignature
	
	
	%%%%%%%%%%%%%%%%%%%
	%Abstract / Kurzzusammenfassung
	%%%%%%%%%%%%%%%%%%%
	%\include{chapters/zusammenfassung}
	
	%%%%%%%%%%%%%%%%%%%
	%Inhaltsverzeichnis 
	%%%%%%%%%%%%%%%%%%%
	\cleardoublepage
	\tableofcontents % Erstellte ein Inhaltsverzeichnis
	
	%\cleardoublepage
	\pagenumbering{arabic} % Seitenzahlen angezeigt, startet ab dem Inhaltsverzeichnis
	\setcounter{page}{1} % Setzt den Seitenzahlenzähler auf 1
	
	%%%%%%%%%%%%%%%%%%%%%%%%%%%%%%%%%%%%%%%%%%%%%%%%%%%%%%%%%%%%%%%%%%%%%%%%%%%%%%%%%%%%%%%%%%%%%%%%%%
	
	% INHALT, am Besten ausgelagert in eigene Files/Kapitel und dann mit \include{Unterordner/Filename} eingefügt, sorgt für bessere Übersichtlichkeit und Fehlersuche. Einzelne Dateien sind aktuell im Ordner Sections abgelegt. 
	
	
	%%%%%%%%%%%%%%%%%%Haupteil%%%%%%%%%%%%%%%%%%%
	
	%%%%%%%%%%%%%%%%%%%%%%%%%%%%%%%%%%%%%%%%%%%%
	
	%%%%%%%%%%%%%%%%%%%%%%%%%%%%%%%%%%%%%%%%%%%%%%%%%%%%%%%%%%%%%%%%%%%%%%%%%%%%%%%%%%%%%%%%%%%%%%%%%%
	
	%%%%%%%%%%%%%%%%%%%
	%Abbildungs- und Tabellenverzeichnis
	%%%%%%%%%%%%%%%%%%%
	\listoffigures % Abbildungsverzeichnis (captions in den Figuren werden als Referenz genommen)
	\listoftables % Verzeichnis der Tabellen (captions in den Tabellen werden als Referenz genommen)
	
	%%%%%%%%%%%%%%%%%%%
	%Literaturverzeichnis an dieser Stelle
	%%%%%%%%%%%%%%%%%%%
	
	
\end{document}
