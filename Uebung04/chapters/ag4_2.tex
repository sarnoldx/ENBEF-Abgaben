\section{Implizites Euler Verfahren}\label{sec:ag4_2}

Zur numerischen Berechnung einfacher Differenzialgleichungen kann das implizite Euler Verfahren
\begin{equation}
\underbrace{
	\left( \dfrac{1}{h} \mathbf{M} + \mathbf{K} \right)
}_{\coloneqq \mathbf{A}}
\mathbf{x_i} =
\underbrace{
	\dfrac{1}{h} \mathbf{M} \mathbf{x_{i-1}} + \mathbf{r}
}_{\coloneqq \mathbf{b_{i-1}}}
\end{equation}

verwendet werden. Hierzu wurde die Methode \texttt{odebwe\_simple} (siehe Listing:\ref{list:odebwe_simple}) in Octave implementiert. Zurückgegeben wird eine Lösungsmatrix $\mathbf{x}$, in der als Spaltenvektoren die verschiedenen Lösungen in Abhängigkeit der Zeit $t$ enthalten sind.

Zum Test der Methode soll nun das Dahlquist-Problem 
\begin{equation}
	\label{gl:dahlquist}
	x^{\prime}(t) = -kx(t)
\end{equation}
für $k=1/5$ und $x(t_0) = 2$ berechnet werden. Die analytisch exakte Lösung ist in diesem Fall durch die Funktion $x(t) = 2 e^{-k t}$ beschrieben. Mithilfe des Skripts Listing \ref{list:dahlquist} wurde die analytische Lösung mit der numerischen verglichen (siehe Abbildung \ref{fig:dahlquist}):

\begin{figure}[h]
	\centering
	\includegraphics[width=0.65\textwidth]{data/dahlquist}
	\caption{\centering Numerische und analytische Lösung der Gleichung \ref{gl:dahlquist}}
	\label{fig:dahlquist}
\end{figure}

Die Methode \texttt{odebwe\_simple} scheint korrekt zu funktionieren.
\newpage
Zuletzt soll das Koaxial Kabel aus Aufgabe 3.1 (Abbildung:\ref{fig:koaxial}) des letzten Übungsblatts erneut berechnet werden. 

\begin{figure}[h]
	\centering
	\includegraphics[width=0.55\textwidth]{data/koaxialkabel1}
	\caption{\centering Ersatzschaltbild eines Koaxialkabels}
	\label{fig:koaxial}
\end{figure}

Zum Lösen der Differenzialgleichung wird im Skript (Listing:\ref{list:lösung3.1}) die zuvor implementierte Methode \texttt{odebwe\_simple} verwendet. Durch Plotten der dritten Zeile der Lösungsmatrix $\mathbf{x}$ erhält man den zeitlichen Verlauf der Spannung am Lastwiderstand $R_f$. Das sich ergebene Bild (Abbildung:\ref{fig:lösung3.1}) gleicht den Ergebnissen des letzten Übungsblatts.

\begin{figure}[h]
	\centering
	\includegraphics[width=0.55\textwidth]{data/lösung3.1}
	\caption{\centering Spannung an Lastwiderstand $R_f$ (siehe Abbildung:\ref{fig:koaxial})}
	\label{fig:lösung3.1}
\end{figure}