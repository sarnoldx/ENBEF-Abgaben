\chapter{Anhang}\label{sec:anhang}
\lstset{ % Octave Settings
	language=Octave,
	extendedchars=true,
	basicstyle=\footnotesize,
	numbers=left,
	numberstyle=\tiny\color{gray},
	stepnumber=1,
	numbersep=10pt,
	showspaces=false,
	showstringspaces=false,
	tabsize=2,
	breaklines=true,
	frame=single,
	morecomment = [l][\itshape\color{blue}]{\%},
	captionpos=b,
	title=\lstname
}
\subsection*{Methode odebwe\_simple}
\begin{lstlisting}[caption={Methode \texttt{odebwe\_simple} in Octave}, label=odebwe_simple]
function x =  odebwe_simple(t,x0,M,K,r)
	% initialize solution vector
	x = zeros(length(x0),length(t));
	x(:,1) = x0;
	
	for i = 2:length(t)
		% time step size
		h = t(i) - t(i-1);
		% system matrix
		A = (1/h)*M + K;
		% right hand side
		b = (1/h)*M*x(:,i-1)+r;
		% solve the linear system Ax=b
		x(:,i) = A\b;
	end
	x
endfunction
\end{lstlisting}

\subsection*{Dahlquist Problem}
\begin{lstlisting}[caption={Lösen des Dahlquist-Problem $x^{\prime}(t) = -kx(t)$ mit $k=1/5$ in Octave}, label=Dahlquist-Problem]
h = 1e-1;
t = 0:h:5;
k = 1/5;
xana =@(t) (2*exp(-k*t));
% function x =  odebwe_simple(t,x0,M,K,r)
xnum = odebwe_simple(t,2,1,k,0);
plot(t,xana(t),t,xnum,'+');
legend('analytisch','numerisch');
xlabel('t-Achse');
ylabel('x-Achse');
\end{lstlisting}
\newpage

\subsection*{Simulation Koaxialkabel}
\begin{lstlisting}[caption={Nummerische Lösung in Octave der bereits berechneten Aufgabe 3.1 a), nun aber mit eigener Methode \texttt{odebwe\_simple}.}, label=Lösung3.1]
t = [0:1e-8:1e-3];

x0 = zeros(5,1);
x0(1) = 12;
x0(2) = 12;

M = zeros(5,5);
M(3,3) = 1e-7; 
M(4,4) = 25e-5;

K = [1000 -1000 0 0 1; -1000 1000 0 1 0; 0 0 1/750 -1 0; 0 -1 1 0 0; -1  0 0 0 0];

r = zeros(5,1);
r(5) = -12;

% Funktion ist deutlich langsamer als die zuvor verwendeten Methoden
x = odebwe_simple(t,x0,M,K,r);

plot(t,x(3,:));
xlabel('Zeit t in [s]', 'interpreter', 'tex')
ylabel('Spannung in [V]', 'interpreter', 'tex')
legend('3. Zeile der Loesungsmatrix x')
\end{lstlisting}