\chapter{Fazit}\label{sec:fazit}
%\addcontentsline{toc}{section}{Fazit}
Die erste Aufgabe ergab, dass es beim Vergleich numerischer Differenzquotienten eine optimale Schrittweise gibt, bei der der Fehler minimal ist. Nach dieser optimalen Schrittweite wächst der Fehler linear mit der selben Steigung wie die Konvergenz der numerischen Approximation.
Die zweite Aufgabe hat gezeigt, dass das Lösen von einfachen Differentialgleichungen mit Hilfe einer selbstgeschriebenen Methode auf Basis des Euler-Verfahrens möglich ist. Es ist dabei aber zu beachten, dass diese Methode erheblich langsamer ist. In der dritten Aufgabe wurden verschiedene Verfahren angewendet um lineare Gleichungssysteme zu lösen. Die Verfahren wurden anhand ihrer Rechenzeit bewertet und außerdem mit Matrizen und rechten Seiten getestet. Der Vergleich ergab, dass das Verfahren, das die Berechnung mit Hilfe von Inversen durchführt das Langsamste ist. Der Backslash Operator benötigt weniger Zeit, ist dennoch nicht so schnell wie die LUPQ-Zerlegung die am wenigste Zeit braucht um Gleichungssysteme zu lösen.
