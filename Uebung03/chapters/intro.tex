\chapter{Einleitung}\label{sec:intro}
%\addcontentsline{toc}{section}{Einleitung}
Diese Arbeit beschäftigt sich mit dem Übungsblatt 3 des Faches \glqq Einführung in die numerische Berechnung elektromagnetischer Felder\grqq{}. In der ersten der drei Aufgaben wird ein Koaxialkabel simuliert und die anliegende Spannung geplottet. Des Weiteren werden durch Anpassungen der Spannung und des Schaltbildes Veränderungen im Frequenzbereich simuliert und untersucht. 
In der zweiten Aufgabe wird eine Differentialgleichungen erst im Zeitbereich und dann im Frequenzbereich gelöst. Die gegebene Schaltung wird zunächst mit LTSpice simuliert und dann mit der Lösung der vorherigen Differentialgleichung analytisch gelöst.
In der letzten Aufgabe wird das elektrische Feld innerhalb eines Plattenkondensator mit dem Betrachtungspunkt der Kapazität in FEMM simuliert. Das Ergebnis der Simulation wird mit der analytischen Lösung verglichen. Abschließend wird im Simulationsprogramm \glqq OctaveFEMM\grqq{} ein Programm entwickelt, dass die Kapazität eines Plattenkondensators mit gegebenen Abmaßen automatisch ermittelt.