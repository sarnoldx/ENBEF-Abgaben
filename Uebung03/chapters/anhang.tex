\chapter{Anhang}\label{sec:anhang}
\lstset{ % Octave Settings
	language=Octave,
	extendedchars=true,
	basicstyle=\footnotesize,
	numbers=left,
	numberstyle=\tiny\color{gray},
	stepnumber=1,
	numbersep=10pt,
	showspaces=false,
	showstringspaces=false,
	tabsize=2,
	breaklines=true,
	frame=single,
	morecomment = [l][\itshape\color{blue}]{\%},
	captionpos=b,
	title=\lstname
}
\subsection*{Methode circuit\_matrices}
\begin{lstlisting}[caption={Methode \texttt{circuit\_matrices} in Octave}, label=circuit_matrices]
function [AC,AL,AR,AV,AI,C,L,R,V,I] = circuit_matrices(filename)

% PEMCE Uebung 2: Routine um Schaltunsmatrizen zu erzeugen:
%  [AC,AL,AR,AV,AI,C,L,R,V,I] = circuit_matrices(filename)
%
% Eingabe:
%  filename  - Dateiname der Netzliste
%
% Ausgaben:
%  AC,...,AI - reduzierte Inzidenzmatrizen fuer Kapazitaten, ..., Stromquellen
%  C,...,I   - Matrizen der Parameter (Kapazitaten, ..., Stromquellen)

%# lese Schaltung aus netzliste
cirstruct = prs_spice (filename);

%# initialisiere variablen
n = cirstruct.totextvar;
AC=AL=AR=AV=AI=sparse(n,0);
C=L=R = zeros(0,0);
V=I= zeros(0,1);

% Durchlaufe alle Elementtypen (R,L,C,...)
for i=1:size(cirstruct.LCR,2)
typ=cirstruct.LCR(i).parnames{1,1};
sz=size(cirstruct.LCR(i).pvmatrix,1);

% Durchlaufe alle Bauteile eines Typen
for j=1:size(cirstruct.LCR(i).pvmatrix,1)
val=cirstruct.LCR(i).pvmatrix(j,:);
pins=cirstruct.LCR(i).vnmatrix(j,:);

if typ == 'L'
matrix = AL;
diagonal = L;
endif 
if typ == 'R'
matrix = AR;
diagonal = R;
endif 
if typ == 'C'
matrix = AC;
diagonal = C;
endif 
if typ == 'V'
matrix = AV;
diagonal = V;
endif  
if typ == 'I'
matrix = AI;
diagonal = I;  
endif

is_voltage_or_current = typ == 'V' || typ == 'I';

matrix = [matrix,sparse(n,1)]
%Diagonalmatrix Elemente einfuegen

if is_voltage_or_current == false
diagonal = [diagonal,zeros(sz,1)];

if typ == 'R'
val = 1/val;
endif

diagonal(j,j) = val;
endif

%Richtung des Stroms bestimmen
direction = pins(1) - pins(2);

%Strom fliesst von 1 nach 2
if direction < 0
%Nur hinzufuegen wenn es sich nicht um den Masseknoten handelt
if pins(1) != 0
matrix(pins(1),j)=1;
endif
matrix(pins(2),j)=-1;

if is_voltage_or_current == true
diagonal = [diagonal;val];
endif

%Strom fliesst von 2 nach 1
else
matrix(pins(1),j)=1;
%Nur hinzufuegen wenn es sich nicht um den Masseknoten handelt
if pins(2) != 0
matrix(pins(2),j)=-1;
endif

if is_voltage_or_current == true
diagonal = [diagonal;-val];
endif
endif

if typ == 'L'
AL = matrix;
L = diagonal;
endif 
if typ == 'R'
AR = matrix;
R = diagonal;
endif  
if typ == 'C'
AC = matrix;
C = diagonal;
endif  
if typ == 'V'
AV = matrix;
V = diagonal;
endif 
if typ == 'I'
AI = matrix;
I = diagonal;  
endif

endfor
endfor
endfunction
\end{lstlisting}

\subsection*{Methode calculate\_matrices}
\begin{lstlisting}[caption={Methode \texttt{calculate\_matrices} in Octave}, label=calculate_matrices]
function [M,K,r] = calculate_matrices(filename)

%Berechnet die Matrizen M,K und r  
[AC,AL,AR,AV,AI,C,L,R,V,I] = circuit_matrices(filename);
bl = size(L,1);
bv = size(V,1);
n = size(AR,1);

M = [AC*C*AC' zeros(n,bl) zeros(n,bv);
zeros(bl,n) L zeros(bl,bv);
zeros(bv,n) zeros(bv,bl) zeros(bv,bv)];

K = [AR*R*AR' AL AV;
-AL' zeros(bl,bl) zeros(bl,bv);
-AV' zeros(bv,bl) zeros(bv,bv)];

r = [-AI*I; zeros(bl,1); -V];
endfunction
\end{lstlisting}

\subsection*{Finales Skript zur Berechnen der Spannung an Rf (n=1)}
\begin{lstlisting}[caption={\texttt{FinalesSkript} in Octave}, label=finalesSkript]
t = [0:1e-8:1e-3];

% Bestimmung des Gleichungssystems:
[M,K,r] = calculate_matrices('koaxialKabel.net');

% r(5) (Spannungsquelle) wird wohl mit falschem Vorzeichen berechnet, diese Korrektur fuehrt nun zum richtigen Ergebnis
r(5) = -r(5);

%res = @(y, yd,t) (M*yd+K*y-r);
res =@(x,xdot,t)(M*xdot+K*x-r);

% Bestimmung der Anfangswerte:
U_q = 12;
L1 = 0.00025;
R1 = 0.001;
C1 = 0.0000001;
R2 = 1500;
Rf = 1500;

% x0:
% phi_1 = 12
% phi_2 = 12
% phi_3 = 0
% i_L   = 0
% i_V   = 0
x0 = zeros(size(r));
x0(1) = U_q;
x0(2) = U_q

% x0_Strich: 
x0_Strich = zeros(size(r));

% Nummerische Berechnung des Gleichungssystems:
x = daspk(res,x0,x0_Strich,t);

% Erstellen des Plots:
plot(t,x(:,3:3));
xlabel('Zeit t in [s]', 'interpreter', 'tex')
ylabel('Spannung in [V]', 'interpreter', 'tex')
legend('\phi_3 entspricht Spannung an Rf')
\end{lstlisting}

\subsection*{Netzliste für n=10}
\begin{lstlisting}[caption={\texttt{Netzliste für Koaxialkabel mit 10 Segmenten} in Octave}, label=netlist10]
Vs A1N1 0 12
Ro1 A1N1 A1N2 0.001
L1 A1N2 A1N3 0.00025
Ru1 0 A1N3 1500
C1 A1N3 0 0.0000001

Ro2 A1N3 A2N2 0.001
L2 A2N2 A2N3 0.00025
Ru2 0 A2N3 1500
C2 A2N3 0 0.0000001

Ro3 A2N3 A3N2 0.001
L3 A3N2 A3N3 0.00025
Ru3 0 A3N3 1500
C3 A3N3 0 0.0000001

Ro4 A3N3 A4N2 0.001
L4 A4N2 A4N3 0.00025
Ru4 0 A4N3 1500
C4 A4N3 0 0.0000001

Ro5 A4N3 A5N2 0.001
L5 A5N2 A5N3 0.00025
Ru5 0 A5N3 1500
C5 A5N3 0 0.0000001

Ro6 A5N3 A6N2 0.001
L6 A6N2 A6N3 0.00025
Ru6 0 A6N3 1500
C6 A6N3 0 0.0000001

Ro7 A6N3 A7N2 0.001
L7 A7N2 A7N3 0.00025
Ru7 0 A7N3 1500
C7 A7N3 0 0.0000001

Ro8 A7N3 A8N2 0.001
L8 A8N2 A8N3 0.00025
Ru8 0 A8N3 1500
C8 A8N3 0 0.0000001

Ro9 A8N3 A9N2 0.001
L9 A9N2 A9N3 0.00025
Ru9 0 A9N3 1500
C9 A9N3 0 0.0000001

Ro10 A9N3 A10N2 0.001
L10 A10N2 A10N3 0.00025
Ru10 0 A10N3 1500
C10 A10N3 0 0.0000001

Rf 0 A10N3 1500
\end{lstlisting}

\subsection*{Finales Skript Vergleich n=1 und n=10}
\begin{lstlisting}[caption={\texttt{SkriptSegmentVergleich1vs10} in Octave}, label=finalesSkript10]
t = [0:1e-8:2e-3];

% Bestimmung des Gleichungssystems:
[M,K,r] = calculate_matrices('koaxialKabel.net');

%r(5) (Spannungsquelle) wird wohl mit falschem Vorzeichen berechnet
r(5) = -r(5);

%res = @(y, yd,t) (M*yd+K*y-r);
res =@(x,xdot,t)(M*xdot+K*x-r);

% Bestimmung der Anfangswerte:
U_q = 12;
L1 = 0.00025;
R1 = 0.001;
C1 = 0.0000001;
R2 = 1500;
Rf = 1500;

% x0:
% phi_1 = 12
% phi_2 = 12
% phi_3 = 0
% i_L   = 0
% i_V   = 0
x0 = zeros(size(r));
x0(1) = U_q;
x0(2) = U_q

% x0_Strich: 
x0_Strich = zeros(size(r));

% Nummerische Berechnung des Gleichungssystems:
% x = dassl(res, transpose(x0), zeros(size(r)), t);
x1 = daspk(res,x0,x0_Strich,t);


%-----------------------------------------------
% Bestimmung des Gleichungssystems mit 10 Segmenten:
%-----------------------------------------------
[M10,K10,r10] = calculate_matrices('KK10.net');

%Spannungsquelle wird wohl mit falschem Vorzeichen berechnet
r10 = -r10;

%res = @(y, yd,t) (M*yd+K*y-r);
res10 =@(x,xdot,t)(M10*xdot+K10*x-r10);

% Bestimmung der Anfangswerte:
U_q = 12;
L1 = 0.00025;
R1 = 0.001;
C1 = 0.0000001;
R2 = 1500;
Rf = 1500;

% x0:
% phi_1 = 12
% phi_2 = 12
% phi_3 = 0
% i_L   = 0
% i_V   = 0
x0_10 = zeros(size(r10));
x0_10(1) = U_q;
x0_10(2) = U_q

% x0_Strich: 
x0_10_Strich = zeros(size(r10))

% Nummerische Berechnung des Gleichungssystems:
% x = dassl(res, transpose(x0), zeros(size(r)), t);
x10 = daspk(res10,x0_10,x0_10_Strich,t);

%----------------------------------------------------
% Erstellen des Plots:
plot(t,x1(:,3:3));
hold on;
plot(t,x10(:,21:21));
hold off;
xlabel('Zeit t in [s]', 'interpreter', 'tex')
ylabel('Spannung in [V]', 'interpreter', 'tex')
legend('Ausgangsspannung bei n=1', 'Ausgangsspannung bei n=10')

\end{lstlisting}