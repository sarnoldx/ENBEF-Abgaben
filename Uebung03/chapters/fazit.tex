\chapter{Fazit}\label{sec:fazit}
%\addcontentsline{toc}{section}{Fazit}
Die erste Aufgabe ergab, dass sich die beiden Leiter des Koaxialkabels wie die Platten eines Plattenkondensators verhalten. Darüber hinaus ergibt sich, dass man durch Anfügen von weiteren Segmenten an die Schaltung eine Verkleinerung der Schwingfrequenz bewirkt.
Differentialgleichungen können häufig, wie sich in Aufgabe zwei zeigt, leichter im Frequenzbereich als im Zeitbereich gelöst werden. Die durch Lösen der Differentialgleichung analytisch berechneten Ergebnisse für Zeit- und Frequenzverhalten stimmen dabei mit der numerischen Simulation durch LTSpice überein.
Die Ergebnisse der dritten Aufgabe ergeben, dass sich die Feldlinien eines Kondensators in einem Simulationskäfig nicht nur senkrecht zu den Platten bewegen, sondern dass sich auch Randeffekte an den Enden der Kondensatorplatten ausbilden. Untersucht man unterschiedliche Randbedingungen zeigt sich, dass diese sowohl den Kapazitätswert des Kondensators, als auch die elektrischen Feldlinien beeinträchtigen. Die Wahl der Simulationsrandbedingungen kann also nicht willkürlich erfolgen.