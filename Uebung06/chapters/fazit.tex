\chapter{Fazit}\label{sec:fazit}
%\addcontentsline{toc}{section}{Fazit}
Die erste Aufgabe ergibt, dass die Stromdichte stetig ist und sich die Ladungsdichte nicht zeitlich verändert.  
Die Ergebnisse der zweiten Aufgabe zeigen, dass bei einem Modell eines Überspannungsableiters ohne Randbedingungen mit zunehmender Höhe im betrachteten Bereich bedeutend stärker ansteigt als der Überspannungsleiter mit Randbedingungen mit dem er verglichen wurde. Darüber hinaus zeigt sich, dass sich bei Betrachtung des feldsteuernden Rings, mit größerem Abstand zwischen Mast und Ring die elektrischen Felder disparat zueinander verändern.