\section{Anhang}
\lstset{ % Octave Settings
	language=Octave,
	extendedchars=true,
	basicstyle=\footnotesize,
	numbers=left,
	numberstyle=\tiny\color{gray},
	stepnumber=1,
	numbersep=10pt,
	showspaces=false,
	showstringspaces=false,
	tabsize=2,
	breaklines=true,
	frame=single,
	morecomment = [l][\itshape\color{blue}]{\%},
	captionpos=b,
	title=\lstname
}

\subsection*{Implementation des Ritz-Galerkin Verfahren}
\begin{lstlisting}[caption={Routine \texttt{make\_Kf} in Octave zur Berechnung der Matrizen $K$ und $f$}, label=code]
function [K,f] = make_Kf(x,a,b)
	% Dimension von x muss eins groesser sein als von a,b
	dim = length(x);
	% ci ist die Konstante vor Matrix Ki
	% di ist die Konstante vor Vektor fi
	for i = 1:(dim-1)
		ci(i,1) = a(i)/(x(i+1)-x(i));
		di(i,1) = 1/2*b(i)*(x(i+1)-x(i));
	endfor
	K = zeros(dim,dim);
	f = zeros(dim,1);
	for i = 1:(dim-1);
		Ki = [1,-1;-1,1];
		Ki = ci(i)*Ki;
		% 2x2 Matrix Ki in grosse Matrix K uebertragen
		K(i,i) = K(i,i) + Ki(1,1);
		K(i+1,i) = Ki(2,1);
		K(i,i+1) = Ki(1,2);
		K(i+1,i+1) = Ki(2,2);
		% Werte in k eintragen
		f(i) = f(i) + di(i);
		f(i+1) = di(i);
	endfor
endfunction
\end{lstlisting}

