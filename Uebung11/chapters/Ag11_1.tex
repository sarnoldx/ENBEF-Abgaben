\chapter{Ausarbeitung der Aufgaben}
\section{Konkrete FEM Matrizen in 1D}
Das Problem $\dfrac{\mathrm{d}}{\mathrm{d}x}\left(a(x)\dfrac{\mathrm{d}}{\mathrm{d}x}U(x)\right)+b(x)=0$ soll mit dem Ritz-Galerkin Verfahren gelöst werden. Hierbei gilt nach der Vorlesung
\begin{equation}
\label{gl:1} 
\mb{K}^i\mb{u}^i=\mb{f}^i,
\end{equation}
\begin{equation} 
\mb{K}^i = \int\limits_{x_i}^{x_{i+1}} \frac{\mathrm{d}(\mb{N}^i(x))^T}{\mathrm{d}x} a(x) \frac{\mathrm{d}\mb{N}^i(x)}{\mathrm{d}x} \ \mathrm{d}x,
\end{equation}
\begin{equation} 
\mb{f}^i = \int\limits_{x_i}^{x_{i+1}} (\mb{N}^i(x))^T b(x) \ \mathrm{d}x,
\end{equation}
mit $a(x)=a_i$ für $x\in\left(x_i,x_{i+1}\right]$ und $b(x)$ analog. Zu dem ist $\mb{N}$ definiert als $\mb{N}^i(x) = \left[N_i(x),N_{i+1}(x)\right]$ mit 
\begin{equation} 
N_i = \begin{cases}
0 & x<x_{i-1} \\
\frac{x-x_{i-1}}{x_i-x_{i-1}} & x_{i-1} \leq x < x_i \\
1-\frac{x-x_i}{x_{i+1}-x_i} & x_i \leq x < x_{i+1} \\
0 & x\geq x_{i+1} 
\end{cases}.
\end{equation}

Die expliziten Formen von $\mb{K}^i$ und $\mb{f}^i$ lauten:
\begin{equation} 
\mb{K}^i = \dfrac{a_i}{x_{i+1}-x_i}
\left[\begin{matrix}
1 & -1  \\
-1 & 1  \\
\end{matrix}\right],
\end{equation}
\begin{equation} 
\mb{f}^i = \dfrac{1}{2}b_i(x_{i+1}-x_i)
\left[\begin{matrix}
1  \\
1  \\
\end{matrix}\right]
\end{equation}

Für den Fall von $n=5$ Knoten ($x_1,x_2,\ldots,x_5$) müssen für jedes Intervall die Matrizen $\mb{K}^i$ und $\mb{f}^i$ aufgestellt werden, also insgesamt $n-1=4$ mal. Das Gleichungssystem (\ref{gl:1}) ergibt sich aus der Zusammensetzung von vier Matrizen $\mb{K}^i$ und $\mb{f}^i$:
 
\begin{equation}
\label{gl:fertig} 
\left[\begin{matrix}
K^1_{1,1} & K^1_{1,2} & 0 & 0 & 0  \\
K^1_{2,1} & K^1_{2,2}+K^2_{1,1} & K^2_{1,2} & 0 & 0  \\
0 & K^2_{2,1} & K^2_{2,2}+K^3_{1,1} & K^3_{1,2} & 0  \\
0 & 0 & K^3_{2,1} & K^3_{2,2}+K^4_{1,1} & K^4_{1,2}  \\
0 & 0 & 0 & K^4_{2,1} & K^4_{2,2}  \\
\end{matrix}\right]
\left[\begin{matrix}
u_1  \\
u_2  \\
u_3  \\
u_4  \\
u_5  \\
\end{matrix}\right]
=
\left[\begin{matrix}
\mathrm{f}^1_1  \\
\mathrm{f}^1_2+\mathrm{f}^2_1  \\
\mathrm{f}^2_2+\mathrm{f}^3_1  \\
\mathrm{f}^3_2+\mathrm{f}^4_1  \\
\mathrm{f}^4_2  \\
\end{matrix}\right]
\end{equation}

Hierbei gilt:
\begin{equation*} 
\mb{K}^i =
\left[\begin{matrix}
K^i_{1,1} & K^i_{1,2}  \\
K^i_{2,1} & K^i_{2,2}  \\
\end{matrix}\right]\text{und }
\mb{f}^i =
\left[\begin{matrix}
\mathrm{f}^i_1  \\
\mathrm{f}^i_2  \\
\end{matrix}\right].
\end{equation*}

Die Routine, die die Matrizen für das Gleichungssystem (\ref{gl:fertig}) aufstellt ist im Anhang zu finden (Code (\ref{code})).